\documentclass{ShadowTeXSR5}
\title{Kleinwildjagd}
%Main author
\author{Daniel}
%\splashpicture{Splash.jpg}
%
\begin{document}
\selectlanguage{ngerman}
% \srmaketitle
%This builds and prints the cover page, splash page and impressum.
% \tableofcontents
% \chapter*{Kleinwildjagd}
\twocolimage{/mnt/chromeos/GoogleDrive/MyDrive/Shadowrun/Sessions/2023-04-17 Kleinwildjagd 3/andreas-aas-schroth-sr5-dpadl-flickenland-rauesland-hshorizontal.jpg}
\hypertarget{kleinwildjagd-1-3}{%
\section{Kleinwildjagd 1-3}\label{kleinwildjagd-1-3}}

\hypertarget{das-treffen-mit-herrn-schmidt}{%
\subsection{Das Treffen mit Herrn Schmidt}\label{das-treffen-mit-herrn-schmidt}}
\index{Subsection}
Euer neuer Schieber Reno Pyatt (der Ex-Kombatbiker-Zwerg mit der leicht amerikanisch angehauchten Bar in Duisburg) hat euch kurzfristig einen Auftrag in Gelsenkirchen verschafft. Der Treffpunkt dazu ist am
\textbf{Kiosk am Loch} in einer extrem heruntergekommen Gegend, die fast schon einem Slum ähnelt. Verfallene oder illegal bewohnte Häuser beherbergen Squatter und einige glückliche Tagelöhner, die bei Kons durch körperliche Arbeit bis zur Altersarmut glänzen dürfen. Müll liegt auf den Kantsteinen, viele Fenster der anliegenden Häuser sind vernagelt. Dornengestrüpp wuchert zwischen den Grundstücken und wilde Tiere huschen gelegentlich umher - ein Bild, das zur Plexweiten Rattenplage durchaus passt. Der Kiosk ist der Treffpunkt der Bewohner des Viertels, die sonst nicht wissen, wohin sie sollen.
Betreiber des Kiosks ist ein Herr \textbf{Björn Damm}, der euch und eure Herrn Schmidt freundlich mit Eintopf und ``hauseigenem'' (vermutlich gepanschtem, aber dafür billigem) Bier bedient. Die Herrn Schmidt machen keinen Hehl aus ihrer Identität oder ihrem Arbeitgeber und es ist offensichtlich, dass sie zum ersten Mal Shadowrunner beschäftigen: es handelt sich um \textbf{Ulrich Hopfenheber} (Elf) und \textbf{Jeremy Basler} (Mensch), die für die Firma \textbf{Engelmann und Kissinger} (kurz ENKI) arbeiten - ein Forschungsunternehmen, das Experimente an
\href{https://shadowhelix.de/Teufelsratte}{\textbf{Teufelsratten}} durchführt.
\twocolimage{/mnt/chromeos/GoogleDrive/MyDrive/Shadowrun/Sessions/2023-04-17 Kleinwildjagd 3/kiosk_am_loch.png}

\hypertarget{der-auftrag}{%
\subsection{Der Auftrag}\label{der-auftrag}}
Die Herren Schmidt erzählen offen, dass sie für ENKI arbeiten und durch den Unfall eines Lieferwagens mit Labormaterial ein so großer Schaden an ihren Gehegen entstanden ist, dass acht der Tiere, eine Art toxischer Ratten, jedoch etwas kleiner als die zurzeit verbreitete Plage an Teufelsratten, entkommen sind. Der zu vergebende Auftrag besteht darin, diese Tiere zu suchen und zurückzubringen. Je schneller die Runner das bewerkstelligen können und je weniger Leute davon mitbekommen, desto höher fällt die Bezahlung aus.
Wenn die Runner die Tiere innerhalb der nächsten 24 Stunden zurückbringen können, erhalten sie als Belohnung 8000 Euro. Brauchen sie bis zu 24 Stunden länger, verringert sich der Lohn auf 5000 Euro. Danach gibt es nur noch 3000 Euro, und zwar ausschließlich unter der Voraussetzung, dass bis dahin niemand durch die Ratten zu Schaden gekommen ist und ebenfalls niemand über ihr Entkommen berichtet hat. Wenn die Runner den Auftrag erledigt haben, können sie sich unter einer Kommlinknummer melden, die Hopfenheber ihnen gibt.
Die Ratten können anhand implantierter RFID-Sender eindeutig identifiziert werden. Die Runner brauchen dabei einfach nur die Tiere in Augmented Reality, z.B. mit einer Smartbrille, zu sehen und werden ein entsprechendes Icon eingeblendet bekommen. Folgende Zusätzliche Infos habt ihr erhalten: - Die Ratten müssen nicht lebend zurückgebracht werden - Fokus liegt auf dem Imageerhalt von ENKI: man möchte lieber als der moderne Rattenfänger von Hameln auftreten, anstatt dass Bürger von den eigenen Ratten verletzt werden - Die Ratten können eventuell mit Soja-Nutella angelockt werden. Sie sind handzahm, etwas kleiner als übliche Teufelsratten, das Leben in der Natur nicht gewöhnt und weden den Kontakt zu Menschen suchen, um zu überleben. - Eine Ratte wurde angeblich bereits am naheliegenden Supermarkt gesichtet

\hypertarget{beinarbeit}{%
\subsection{Beinarbeit}\label{beinarbeit}}

\hypertarget{engelmann-kissinger}{%
\subsubsection{Engelmann \& Kissinger}\label{engelmann-kissinger}}
Das Kiosk am Loch liegt auf der gleichen Straße, wie das Labor. Jeremy Basler zeigt euch den Schaden an dem Gebäude, der durch die Explosion eines anliefernden Lasters entstanden ist. Es gibt keine Hinweise auf einen Anschlag und wenigstens mäßige Überwachung innerhalb des Gebäudes. Neben eine Reihe weiterer Critter, gibt es noch Reihen anderer gezüchteter Ratten. Ihr wisst also, wie die entflohenen Teufelsratten aussehen. Von Herrn Basler erhaltet ihr noch drei Fallen mit Matrix-Anbindung, die etwa wie Transportboxen für Katzen oder Hunde aussehen, und dazu eine Ladung Soja-Nutella.

\hypertarget{auf-der-strauxdfe}{%
\subsubsection{Auf der Straße}\label{auf-der-strauxdfe}}
Bei Unterhaltungen mit verschiedenen Personen am Kiosk und auf der Straße zur Brache bzw. zum Labor habt ihr folgendes in Erfahrung gebracht: - Es gibt eine nahegelegene Schule mit einer (angeblichen) Rattenplage - ihr konntet das bisher nicht valideren, sondern konntet nur extremen Einsatz von chemischen Unkrautvernichtern feststellen - In dem alten Kleingartengelände bzw. der Brache gibt es einen ``toxischen Kult'', der angeblich einen Rattengott anbetet und Kinder verspeist. Es gibt aber keine Berichte von fehlenden Kindern. - Am heutigen Abend wird es ein Fußballspiel (Schalke gegen St.Pauli) und Public Viewing am Kiosk geben. - Ebenfalls am Kiosk soll es regelmäßig Critterkämpfe geben (wie bewettete Hahnenkämpfe, mit teilweise magischen Tieren), die in der Nachbarschaft sehr beliegt sind. Das nächste Event soll bereits morgen Abend stattfinden. - Einige Anwohner geben Rattensichtungen an, die aber nicht validiert werden konnten.

\hypertarget{die-bracheehem.-kleingartenverein}{%
\subsubsection{Die Brache/ehem. Kleingartenverein}\label{die-bracheehem.-kleingartenverein}}
\includegraphics[width=\columnwidth, height=0.3\textheight]{/mnt/chromeos/GoogleDrive/MyDrive/Shadowrun/Sessions/2023-04-17 Kleinwildjagd 3/Brache.jpg}
Die ehemalige Kleingartensiedlung in Gelsenkirchen ist seit Jahren eine verseuchte Zone, die von den meisten Leuten gemieden wird. Viele Hinweise deuten darauf hin, dass sich die entkommenen Ratten hier jedoch sehr wohl fühlen würden.

\includegraphics[width=\columnwidth, height=0.3\textheight]{/mnt/chromeos/GoogleDrive/MyDrive/Shadowrun/Sessions/2023-04-17 Kleinwildjagd 3/Brache2.jpg}
Das Dreieck zwischen Schnellstraßen und Industriegeinet ist locker bewaldet und bot früher engagierten Kleingarten-Fans Unterschlupf und Anbaufläche. Hier und da sind noch Reste von Hütten zu erkennen. Ein ausgiebiger Erkunfdungsflug eures Riggers zeigte euch alte, verbogene Zäune und einen großen Parkplatz am Rand des Geländes, auf dem nun verzweifelte Squatter ihre Zelte errichtet haben. Weiter als ein paar Schritte vom Rand entfernt sich allerdings niemand freiwillig. Zu gefährlich könnte sein, was sich im Gebüsch oder den überwucherten Gemüsebeeten befindet. Und manchmal ist es genau dieses alte Gemüse, das jetzt mit Zähnen und Säure zurückschlägt.
Ihr habt hier bereits am Nachmittag hier zwei der drei Fallen (natürlich mit Soja-Nutella bestückt) aufgestellt, in der Hoffnung, dass zeitnah einige der gesuchten Ratten sich anlocken lassen. Da das erst ein paar Stunden her ist haben die Fallen aber noch keinen Erfolg gemeldet. Die dritte Falle habt ihr an der \textbf{naheliegenden Schule} platziert, bei der angeblich kürzlich Teufelsrattenbefall zu Infektionen bei den Schulkindern und Heimunterricht geführt hat.

\hypertarget{der-dreifach-erleuchtete-orden-der-fuxfcnf-wege-des-sterns}{%
\subsubsection{Der dreifach Erleuchtete Orden der fünf Wege des Sterns}\label{der-dreifach-erleuchtete-orden-der-fuxfcnf-wege-des-sterns}}
Bei einer längeren Drohnenaufklärung mit den Flugdrohnen eures Riggers konntet ihr mitten im Kleingartengelände zwei notdürftig reparierte größere Häuser entdecken - ein ehemaliges Vereinshaus und eine Lagerhalle. Das Vereinshaus ist bewohnt. Insgesamt konntet ihr sieben Erwachsene und 2 zwei Kinder (ca 10 Jahre) finden, die sich auf dem Gelände aufgehalten haben. Die Dächer der Gebäude zeigten ein fremdartiges Emblem, das ihr bei einem befreundeten Decker in der Matrix habt suchen lassen.
Die Ergebnisse dieser Matrixsuche ließen unangenehme Vorahnung wahr werden: es scheint sich um den sektenartigen Kult der dreifach Erleuchtete Orden der fünf Wege des Sterns zu handeln. Da es eine relativ kleine Gruppe ist, war es schwierig in der Matrix mehr Details zu bekommen, außer dem Namen der Anführerin: Aina Raabe.
Durch das Expertenwissen eurer eigenen Schamanin wisst ihr außerdem, dass solche Kulte gescheiterter Existenzen, die sich Geistern für angebliche Erlösung oder postapokalyptische Belohnung zuwenden, nicht selten sind. Außer Gerüchten und den gefangenen Tieren in dem Nebengebäude habt ihr aber keine Hinwise auf den Wahn in dieser Gruppe. Ihr konntet aber noch beobachten, dass auf einem freien Feld neben dem Hauptgebäude anscheinend gerade etwas aufgebaut bzw. vorbereitet wird\ldots{} vielleicht ein Ritual?

\hypertarget{aldi-real-die-tote-ratte}{%
\subsubsection{Aldi-Real \& die tote Ratte}\label{aldi-real-die-tote-ratte}}
Bei Aldi-Real habt ihr mit dem zuständigen Manager gesprochen, der am Vorabend den Kammerjäger \textbf{Kodde Fööt} beauftragt hat, eine Ratte vom Anlieferungsparkplatz zu entfernen. Ihr konntet euch bestätigen lassen, dass diese Ratte bereits getötet wurde und sich jetzt beim Kammerjäger in Chorweiler befindet.

\hypertarget{kodde-fuxf6uxf6t-magische-rattenjuxe4ger}{%
\subsubsection{Kodde Fööt: Magische Rattenjäger}\label{kodde-fuxf6uxf6t-magische-rattenjuxe4ger}}
Durch eine schnelle Matrixsuche bei eurem befreundeten Decker habt ihr erfahren, dass Kodde Fööt ein professioneller, nicht billiger Kammerjäger mit ganz gutem Ruf ist. Es werden konventionelle, wie magische Methoden in der Matrixpräsenz der Firma angepriesen. Der Geschäftsführer und Mitgründer ist ein gewisser \textbf{Mikku Kadenberger}.
Zuletzt habt ihr euch am Abend des Tages der Auftragvergabe, gegen 18:00, auf den Weg nach
\href{https://www.google.com/maps/d/viewer?mid=1LAs03Ps-kPhd6v7WgEsD6yqa0XnqBms\&ll=51.020533411852696\%2C6.863011874067486\&z=15}{Chorweiler} gemacht. Direkten Kontakt mit der Firma habt ihr noch nicht aufgenommen. Als ihr vor eingetroffen seid, habt ihr euch mit Drohneneinsatz und astral einen Überblick vom Gelände verschafft, habt bereits die Kameras im Ausbereich geschickt umgangen, um den Müll nach Rattenleichen zu durchsuchen und festgestellt, das eine Reihe erwachter Personen sich auch nach der offiziellen Öffnungszeit im Gebäude aufhält und eine Art Zirkel auf dem Boden des größten Raumes gebildet hat\ldots{}

\newpage
\hypertarget{kleinwildjagd-4}{\section{Kleinwildjagd 4}\label{kleinwildjagd-4}}
\includegraphics[width=\columnwidth]{/mnt/chromeos/GoogleDrive/MyDrive/Shadowrun/Sessions/2023-04-17 Kleinwildjagd 3/mikku_kadenberger.jpg}
\hypertarget{plaudern-mit-mikku}{\subsubsection{Plaudern mit Mikku}\label{plaudern-mit-mikku}}
Neu zu euch gestoßen (aber im Herzen immer dabei gewesen) ist der Runner Fox. Ein modisch gekleideter Zwerg, der vornehm zurückhaltend bei euch im Van platzgenommen hat. Durch seine Kommentare schien bereits durch, dass die magisch erwachte Fraktion eurer illustren kleinen Runde Verstärkung bekommen hat.
Nach einigen technischen Problemen und ausführlicher Sondierung der Lage habt ihr beschlossen, bei der weiteren Verfolgung der Ratten diplomatisch vorzugehen. Jericho hat mithilfe seiner Adeptenkräfte die Form von Jericholinchen - einem unschuldigen (aber fast zweimetergroßem) jungen Mädchen angenommen und einfach mal an der Tür geklingelt.

\newpage
\hypertarget{zusuxe4tzliches-material}{%
\section{Zusatzmaterial}\label{zusuxe4tzliches-material}}
Auf der \href{https://www.google.com/maps/d/edit?mid=1LAs03Ps-kPhd6v7WgEsD6yqa0XnqBms\&usp=sharing}{RRP-Übersichtskarte} könnt ihr die Region in Gelsenkirchen sehen. Hier ein kleiner Überblick über das fragliche Gebiet:
\twocolimage{/mnt/chromeos/GoogleDrive/MyDrive/Shadowrun/Sessions/2023-04-17 Kleinwildjagd 3/karte.jpg}

\end{document}
